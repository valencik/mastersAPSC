\documentclass{beamer}
%\usetheme{Antibes}
%\usetheme{Montpellier}
%\usetheme{Singapore}
%\usetheme{Hannover}
%\usetheme{Malmoe}
\usetheme{Copenhagen}
%\usetheme{Berkeley}

\usecolortheme{beaver}
%\usecolortheme{rose}
%\usecolortheme{dolphin}

 %\usepackage{beamerthemesplit} // Activate for custom appearance
 \setlength{\unitlength}{\textwidth}  % measure in textwidths
 \usepackage{graphicx}

\title{An Improved Vacuum System}
\subtitle{\Large{for Storing Highly Charged Ions in a Penning Trap}}
\author{Damien Robertson - TITAN Summer 2011}
\date{August 24, 2011}

\begin{document}
\frame
{
\titlepage
%\begin{picture}(0.0,0.0)
%     \put(0.0,0.0){\includegraphics[width=0.75\textwidth]{/Users/Damo/Documents/SMU/Year3.2/PHYS3210/assign03/q2b.pdf}}
%\end{picture}
}

\frame
{
  \frametitle{Agenda}
  \begin{itemize}
  \item<1> Introduction to TITAN
  \item<1> Ion Pump Baking and Testing
  \item<1> EIGB Design and Testing
  \item<1> Conclusions
  \end{itemize}
}
%\frame{\tableofcontents}
%%%%%%%%%%%%%%%%%%%%%%%%%%%%%%%%%%%%%%%%%%%%%%%%%%%%%
%%INTRODUCTION%%%%%%%%%%%%%%%%%%%%%%%%%%%%%%%%%%%%%%%%%%
\section{Introduction}
\subsection{The Motivation}
\frame
{
  \frametitle{Motivation}
  \begin{columns}[T]
    \begin{column}{0.450\textwidth}
      \begin{itemize}
        \item<1> Gave valves provide safety in event of vacuum failure.
        \item<1> Interlocks - gave valves close, high voltage power shut off.
        \item<1> Compartments rely on ionization and convection gauges.
        \item<1> Convection gauge operates in mtorr range.
        \item<1> Ion gauge interferes with MCP.
      \end{itemize}
    \end{column}
    \begin{column}{0.550\textwidth}
    \includegraphics[width=1.0\textwidth]{/Users/Damo/Documents/TRIUMF/mpetBeamlineCurrent.pdf}
    \end{column}
  \end{columns}
}
\frame
{
  \frametitle{Ion Gauge Interference}
  \begin{table}[htdp]
  \centering
  \begin{tabular}{cc}
  \includegraphics[width=0.42\textwidth]{/Users/Damo/Documents/TRIUMF/tofSpecIGon.pdf}&\includegraphics[width=0.42\textwidth]{/Users/Damo/Documents/TRIUMF/resonance757.pdf}\\
  \includegraphics[width=0.42\textwidth]{/Users/Damo/Documents/TRIUMF/tofSpecIGoff.pdf}&\includegraphics[width=0.42\textwidth]{/Users/Damo/Documents/TRIUMF/resonance758.pdf} 
  \end{tabular}
  \end{table}%
}
\frame
{
  \frametitle{Charge Exchange}
  \begin{itemize}
  \item<1>Higher pressure translates into greater likelihood that ions may interact with stray molecules.
  \item<1>Changing the charge state of ions is problematic especially for HCI.
  \item<1>Low vacuum and ion gauges can contribute to charge exchange by making the vacuum worse.
  \item<1>Critical that UHV is achieved for mass measurement, MPET.
  \end{itemize}
}
%%Ion Pump Baking and Testing%%%%%%%%%%%%%%%%%%%%%%%%%%%%%%%%%%%%
\section{Ion Pump Baking and Testing}
\subsection{Ion Pump Baking}
\frame
{
  \frametitle{Ion Pump Baking Preparation}
  \begin{columns}[T]
    \begin{column}{0.5\textwidth}
    \includegraphics[width=0.90\textwidth]{/Users/Damo/Documents/TRIUMF/bakingsetup.jpg}
    \end{column}
    \begin{column}{0.5\textwidth}
    \includegraphics[width=0.85\textwidth]{/Users/Damo/Documents/TRIUMF/IMG_0262.jpg}
    \end{column}
  \end{columns}
}
\frame
{
  \frametitle{Ion Pump Baking}
  \begin{table}[htdp]
  \centering
  \begin{tabular}{cc}
  \includegraphics[width=0.47\textwidth]{/Users/Damo/Documents/TRIUMF/bakeTemp3.pdf}&\includegraphics[width=0.47\textwidth]{/Users/Damo/Documents/TRIUMF/bakePressure2.pdf}\\
  \end{tabular}
  \end{table}%
  \begin{itemize}
  \item<1>Start pressure $1.78\times10^{-10}$ torr, end pressure $5.43\times10^{-11}$ torr.
  \end{itemize}
}
\subsection{Ion Pump Testing}
\frame
{
  \frametitle{Ion Pump Operation}
  \begin{columns}[T]
    \begin{column}{0.50\textwidth}
      \begin{itemize}
        \item<1>Large permanent magnets, HV cathode plates.
        \item<1>Honeycombed cylinders maintain plasma that ionizes air molecules.
        \item<1>Ions are accelerated towards cathode plates, are buried or sputter.
        \item<1>Ion pump has three HV settings, 3 kV, 5 kV and 7 kV.
      \end{itemize}
    \end{column}
    \begin{column}{0.50\textwidth}
    \includegraphics[width=1.0\textwidth]{/Users/Damo/Documents/TRIUMF/IonPumpLesker.jpg}
    \end{column}
  \end{columns}
}
\frame
{
  \frametitle{Testing Procedure}
  \begin{itemize}
  \item<1>Ion Pump had not been activated for two years.
  \item<1>Correct starting procedure was observed.
  \item<1>All three settings were tried while manual gate valve was both open and closed.
  \item<1>Pressure allowed to stabilize after setting changed and recorded.
  \item<1>Ran through twice to ensure reproducibility.
  \end{itemize}
}
\frame
{
  \frametitle{Testing Results}
  \begin{table}[htdp]
\begin{center}
\begin{tabular}{|c|c|c|}
\hline
&Gate valve open (torr) &Gate valve closed (torr) \\
\hline
7 kV&$1.0\pm 0.1\times 10^{-10}$&$1.1\pm0.1\times 10^{-10}$\\
5 kV&$5.4\pm0.2\times 10^{-11}$&$9.2\pm0.3\times 10^{-11}$\\
3 kV&$3.8\pm0.1\times 10^{-11}$&$7.2\pm0.2\times 10^{-11}$\\
\hline
\end{tabular}
\end{center}
\label{Table 1}
\end{table}%
  \begin{itemize}
  \item<1> All trials showed ion pump performed better with manual gate valve open.
  \item<1> Based on present test, optimum setting is: Manual gate valve open, HV set to 3 kV.
  \item<1> Optimum setting was used and allowed to pump a further two days, pressure went off scale, $<1.0\times10^{-11}$ torr.
  \end{itemize}
}
%%EIGB Design and Testing%%%%%%%%%%%%%%%%%%%%%%%%%%%%%%%%%%%%%%
\section{EIGB Design and Testing}
\subsection{Concept and Design}
\frame
{
  \frametitle{Theory}
  \begin{columns}[T]
    \begin{column}{0.50\textwidth}
      \begin{itemize}
        \item<1>Ion gauge produces charged particles
        \item<1>Through the use of electrostatic barrier, charged particles can be contained.
      \end{itemize}
    \end{column}
    \begin{column}{0.50\textwidth}
    \includegraphics[width=1.0\textwidth]{/Users/Damo/Documents/TRIUMF/elecPot.pdf}
    \end{column}
  \end{columns}
}
\frame
{
  \frametitle{Design}
  \begin{table}[htdp]
  \centering
  \begin{tabular}{cc}
  \includegraphics[width=0.47\textwidth]{/Users/Damo/Documents/TRIUMF/innerApparatusV5b.pdf}&\includegraphics[width=0.47\textwidth]{/Users/Damo/Documents/TRIUMF/fullApparatusV5bb.pdf}\\
  \end{tabular}
  \end{table}%
  \begin{itemize}
  \item<1>SolidWorks design (EIGB) Electrostatic Ion Gauge Barrier.
  \item<1>Conductive screens act as electrostatic barrier.
  \item<1>Insulating spheres, BalSeal spring maintain electrical continuity.
  \end{itemize}
}
\frame
{
\frametitle{Design}
  \begin{table}[htdp]
  \centering
  \begin{tabular}{cc}
  \includegraphics[width=0.45\textwidth]{/Users/Damo/Documents/TRIUMF/IMG_0177.jpg}&\includegraphics[width=0.45\textwidth]{/Users/Damo/Documents/TRIUMF/IMG_0183.jpg}\\
  \end{tabular}
  \end{table}%
  }
\subsection{Testing}
\frame
{
  \frametitle{Test Setup and Procedure}
  \begin{itemize}
  \item<1> Intended to test in baking station in ISAC 1 Hall.
  \item<1> Test setup would involve multiple scenarios involving ion gauge placement and an MCP. 
  \item<1> The EIGB and ion gauge would be within line of sight of each other, and multiple scenarios in which they would not be within line of sight of each other.
  \item<1>Aim would determine if EIGB is necessary to create barrier for ions or if simply placing the ion gauge around a simple bend, thus eliminating line of sight, would tame the unwanted background noise on the MCP.
  \item<1> Time did not allow such tests.
  \end{itemize}
}
\frame
{
  \frametitle{Testing EIGB in MPET}
  \begin{itemize}
  \item<1>MCP biased to $\approx 1950$ V.
  \item<1>Various scenarios using barrier voltage from $0\--3000$ V.
  \end{itemize}
  \begin{table}[htdp]
\begin{tabular}{|c|c|}
\hline
&MCP count rate (Hz)\\
\hline
Ion gauge in original position&$\approx$15000\\
Ion gauge in EIGB, barrier potentials off&$\approx450$\\
Ion gauge in EIGB, barrier potentials on&$\approx300$\\
\hline
\end{tabular}
\end{table}%
\begin{itemize}
\item<1>Barriers set to $\pm500$ V.
\item<1>EIGB's best performance; middle screen biased to 200 V or higher, front screen biased to $-150$ V or lower.
\item<1>Strange phenomena at 150 V.
\end{itemize}
}
\frame
{
  \frametitle{Results of EIGB in MPET}
  \begin{columns}[T]
    \begin{column}{0.50\textwidth}
      \begin{itemize}
        \item<1>Performance of EIGB was not as expected but new phenomena discovered.
        \item<1>Testing found that ion pump/IG1 can contribute to MCP interference as well.
        \item<1>This interference was thought to be unavoidable background noise.
        \item<1>Raises new questions about filtering interference.
        \item<1>Proper test setup should identify new problems.
      \end{itemize}
    \end{column}
    \begin{column}{0.50\textwidth}
    \includegraphics[width=0.9\textwidth]{/Users/Damo/Documents/TRIUMF/IMG_0261.jpg}
    \end{column}
  \end{columns}
}
%%CONCLUSIONS%%%%%%%%%%%%%%%%%%%%%%%%%%%%%%%%
\section{Conclusions}
\subsection{Ion Pump}
\frame
{
  \frametitle{Ion Pump Implementation}
  \begin{columns}[T]
    \begin{column}{0.450\textwidth}
      \begin{itemize}
        \item<1> Given the success of the ion pump, second pump should be added, shown in light grey.
        \item<1> May provide the lowest pressure for the MPET.
        \item<1> Ion gauge, IG3, added to monitor pressure in MPET, outside of mag field.
      \end{itemize}
    \end{column}
    \begin{column}{0.550\textwidth}
    \includegraphics[width=1.0\textwidth]{/Users/Damo/Documents/TRIUMF/mpetBeamlineBeta2.pdf}
    \end{column}
  \end{columns}
}
\subsection{Electrostatic Ion Gauge Barrier}
\frame
{
  \frametitle{Limitations}
  \begin{itemize}
  \item<1> Removing ion gauge from line of sight showed improvement by a factor of $\approx$ 33.
  \item<1> Activating EIGB showed improvement by a factor of $\approx$ 1.3.
  \item<1> May be other effects of charged particle creation from ion gauge EIGB is missing.
  \item<1> Possible that EIGB is missing charged particles due to convex conductive screens.
  \item<1> Further study needed to improve design of EIGB to activate IG2.
  \end{itemize}
}
% CLOSING-----------------------------------------------------
\subsection{Closing}
\frame
{
  \frametitle{Closing}
  \begin{itemize}
  \item<1> Thank you to Stephan Ettenauer, Usman Chowdhury and Mel Good for their help and the rest of the TITAN team for allowing me to "try" stuff out on MPET.
  \end{itemize}
}
%%END%%%%%%%%%%%%%%%%%%%%%%%%%%%%%%%%%%%%%%%
\end{document}

%TWO COLUMN FRAME W/IMAGE%%%%%%%%%%%%%%%%%%%%
\frame
{
\frametitle{A frame}
  \begin{columns}[T]
    \begin{column}{0.5\textwidth}
      \begin{itemize}
        \item<1-> Normal LaTeX class.
      \end{itemize}
    \end{column}
    \begin{column}{0.5\textwidth}
    \includegraphics[width=1.0\textwidth]{/Users/Damo/Documents/SMU/Year3.2/PHYS3210/assign03/q2c.pdf}
    \end{column}
  \end{columns}
}
%TWO COLUMN FRAME W/IMAGE%%%%%%%%%%%%%%%%%%%%%
%THREE IMAGE FRAME%%%%%%%%%%%%%%%%%%%%%%%%%%
\frame
{
%\frametitle{A frame}
  \begin{picture}(0.0,0.5)
     \put(-0.1,-0.05){\includegraphics[width=0.75\textwidth]{/Users/Damo/Documents/SMU/Year3.2/PHYS3210/project/presentation/HC1proj.pdf}}
  \end{picture}
  \begin{picture}(0.0,0.5)
     \put(0.54,0.23){\includegraphics[width=0.475\textwidth]{/Users/Damo/Documents/SMU/Year3.2/PHYS3210/project/presentation/HC1top.pdf}}
  \end{picture}
  \begin{picture}(0.0,0.5)
     \put(0.529,-0.08){\includegraphics[width=0.475\textwidth]{/Users/Damo/Documents/SMU/Year3.2/PHYS3210/project/presentation/HC1side.pdf}}
  \end{picture}
} 
%THREE IMAGE FRAME%%%%%%%%%%%%%%%%%%%%%%%%%%


